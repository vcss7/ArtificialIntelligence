\documentclass[11pt]{article}

\usepackage{xcolor}
\usepackage{enumitem}
\usepackage[margin=1.0in]{geometry}
\usepackage[none]{hyphenat}
\usepackage{amsfonts,amsmath,amssymb}
\usepackage{fancyhdr}

\thispagestyle{fancy}
\headheight14pt
\fancyhead{}
\fancyhead[L]{\slshape\MakeUppercase{\large Assignment-02: Logic and Logic
    Reasoning}} 
\fancyhead[R]{\large \slshape Victor Solis | \today}

\begin{document}
\begin{flushleft}
Artificial intelligence assignment 2; logic and logic reasoning.

\section*{1. Logic}

Consider the following statements:
    \begin{enumerate}

        \item[a.] $((smoke \land heat \rightarrow fire) \leftrightarrow (smoke \rightarrow fire) \lor (heat \rightarrow fire))$
            \begin{enumerate} 

                \item[i.] Show the truth table for this sentence. \par
                    \medskip
                    \begin{tabular}{@{ }c@{ }@{ }c@{ }@{ }c | c@{ }@{}c@{}@{}c@{}@{ }c@{ }@{ }c@{ }@{ }c@{ }@{}c@{}@{ }c@{ }@{ }c@{ }@{}c@{}@{ }c@{ }@{}c@{}@{}c@{}@{ }c@{ }@{ }c@{ }@{ }c@{ }@{}c@{}@{ }c@{ }@{}c@{}@{ }c@{ }@{ }c@{ }@{ }c@{ }@{}c@{}@{}c@{}@{ }c}
                    S & H & F &  & ( & ( & S & $\land$ & H & ) & $\rightarrow$ & F & ) & $\leftrightarrow$ & ( & ( & S & $\rightarrow$ & F & ) & $\lor$ & ( & H & $\rightarrow$ & F & ) & ) & \\
                    \hline 
                    T & T & T &  &  &  & T & T & T &  & T & T &  & \textcolor{red}{T} &  &  & T & T & T &  & T &  & T & T & T &  &  & \\
                    T & T & F &  &  &  & T & T & T &  & F & F &  & \textcolor{red}{T} &  &  & T & F & F &  & F &  & T & F & F &  &  & \\
                    T & F & T &  &  &  & T & F & F &  & T & T &  & \textcolor{red}{T} &  &  & T & T & T &  & T &  & F & T & T &  &  & \\
                    T & F & F &  &  &  & T & F & F &  & T & F &  & \textcolor{red}{T} &  &  & T & F & F &  & T &  & F & T & F &  &  & \\
                    F & T & T &  &  &  & F & F & T &  & T & T &  & \textcolor{red}{T} &  &  & F & T & T &  & T &  & T & T & T &  &  & \\
                    F & T & F &  &  &  & F & F & T &  & T & F &  & \textcolor{red}{T} &  &  & F & T & F &  & T &  & T & F & F &  &  & \\
                    F & F & T &  &  &  & F & F & F &  & T & T &  & \textcolor{red}{T} &  &  & F & T & T &  & T &  & F & T & T &  &  & \\
                    F & F & F &  &  &  & F & F & F &  & T & F &  & \textcolor{red}{T} &  &  & F & T & F &  & T &  & F & T & F &  &  & \\
                    \end{tabular}
                    \medskip
                \item[ii.] Is this sentence Valid, Satisfiable, or neither? \par
                    This sentence is \textbf{valid}.

            \end{enumerate}
        \item[b.] $(a_1 \lor a_3) \land (\neg a_1 \lor a_2) \land (\neg a_1 \lor a_4) \land (\neg a_1 \lor \neg a_4) \land (\neg a_3)$

            \begin{enumerate} 
                \item[i.] Show the truth table for this sentence. \par
                    \medskip
                    \begin{tabular}{@{ }c@{ }@{ }c@{ }@{ }c@{ }@{ }c | c@{ }@{}c@{}@{}c@{}@{ }c@{ }@{ }c@{ }@{ }c@{ }@{}c@{}@{ }c@{ }@{}c@{}@{ }c@{ }@{ }c@{ }@{ }c@{ }@{ }c@{ }@{}c@{}@{}c@{}@{ }c@{ }@{}c@{}@{}c@{}@{ }c@{ }@{ }c@{ }@{ }c@{ }@{ }c@{ }@{}c@{}@{ }c@{ }@{ }c@{ }@{ }c@{ }@{}c@{}@{ }c}
                    $a_1$ & $a_2$ & $a_3$ & $a_4$ &  & ( & ( & $a_1$ & $\lor$ & $a_3$ & ) & $\land$ & ( & $\lnot$ & $a_1$ & $\lor$ & $a_2$ & ) & ) & $\land$ & ( & ( & $\lnot$ & $a_1$ & $\lor$ & $a_4$ & ) & $\land$ & $\lnot$ & $a_3$ & ) & \\
                    \hline 
                    T & T & T & T &  &  &  & T & T & T &  & T &  & F & T & T & T &  &  & \textcolor{red}{F} &  &  & F & T & T & T &  & F & F & T &  & \\
                    T & T & T & F &  &  &  & T & T & T &  & T &  & F & T & T & T &  &  & \textcolor{red}{F} &  &  & F & T & F & F &  & F & F & T &  & \\
                    T & T & F & T &  &  &  & T & T & F &  & T &  & F & T & T & T &  &  & \textcolor{red}{T} &  &  & F & T & T & T &  & T & T & F &  & \\
                    T & T & F & F &  &  &  & T & T & F &  & T &  & F & T & T & T &  &  & \textcolor{red}{F} &  &  & F & T & F & F &  & F & T & F &  & \\
                    T & F & T & T &  &  &  & T & T & T &  & F &  & F & T & F & F &  &  & \textcolor{red}{F} &  &  & F & T & T & T &  & F & F & T &  & \\
                    T & F & T & F &  &  &  & T & T & T &  & F &  & F & T & F & F &  &  & \textcolor{red}{F} &  &  & F & T & F & F &  & F & F & T &  & \\
                    T & F & F & T &  &  &  & T & T & F &  & F &  & F & T & F & F &  &  & \textcolor{red}{F} &  &  & F & T & T & T &  & T & T & F &  & \\
                    T & F & F & F &  &  &  & T & T & F &  & F &  & F & T & F & F &  &  & \textcolor{red}{F} &  &  & F & T & F & F &  & F & T & F &  & \\
                    F & T & T & T &  &  &  & F & T & T &  & T &  & T & F & T & T &  &  & \textcolor{red}{F} &  &  & T & F & T & T &  & F & F & T &  & \\
                    F & T & T & F &  &  &  & F & T & T &  & T &  & T & F & T & T &  &  & \textcolor{red}{F} &  &  & T & F & T & F &  & F & F & T &  & \\
                    F & T & F & T &  &  &  & F & F & F &  & F &  & T & F & T & T &  &  & \textcolor{red}{F} &  &  & T & F & T & T &  & T & T & F &  & \\
                    F & T & F & F &  &  &  & F & F & F &  & F &  & T & F & T & T &  &  & \textcolor{red}{F} &  &  & T & F & T & F &  & T & T & F &  & \\
                    F & F & T & T &  &  &  & F & T & T &  & T &  & T & F & T & F &  &  & \textcolor{red}{F} &  &  & T & F & T & T &  & F & F & T &  & \\
                    F & F & T & F &  &  &  & F & T & T &  & T &  & T & F & T & F &  &  & \textcolor{red}{F} &  &  & T & F & T & F &  & F & F & T &  & \\
                    F & F & F & T &  &  &  & F & F & F &  & F &  & T & F & T & F &  &  & \textcolor{red}{F} &  &  & T & F & T & T &  & T & T & F &  & \\
                    F & F & F & F &  &  &  & F & F & F &  & F &  & T & F & T & F &  &  & \textcolor{red}{F} &  &  & T & F & T & F &  & T & T & F &  & \\
                    \end{tabular}
                    \medskip
                \item[ii.] Is this sentence Valid, Satisfiable, or neither? \par
                    This sentence is \textbf{satisfiable}.
            \end{enumerate}

    \end{enumerate}

\rule[0.1pt]{40em}{1.0pt}

\section*{2. Vocabulary} First-order logic with consistent vocabulary.

Represent the following sentences in first-order logic, using a consistent
    vocabulary (which you must define as well).

\begin{enumerate}

    \item[a.] Define the vocabulary.
        \begin{enumerate} 
            \item[i.] Buys(x, p) = Person x buys Policy p.
            \item[i.] Expensive(p) = Policy p is expensive.
            \item[iii.] Barber(x) = Person x is a barber.
            \item[iv.] Shaves(x, y) = Person x shaves Person y.
            \item[v.] Born(x, UK) = Person x is born in the UK.
            \item[vi.] Parent(x, y) = Person x is the parent of Person y.
            \item[vii.] BornCitizen(x, UK) = Person x is born in the UK and
                is a citizen of the UK.
            \item[viii.] CitizenByDescent(x, UK) = Person x is a citizen of
                the UK by descent.
            \item[ix.] Politician(x) = Person x is a politician.
            \item[x.] canFool(x, y, t) = Person x fools Person y at time t.
            \item[xi.] Student(x) = Person x is a student.
            \item[xii.] Russian(x, y, z) = Person x took Russian in year y
                in semester z.
            \item[xiii.] German(x, y, z) = Person x took German in year y
                in semester z.
            \item[xiv.] BestScore(x, y) = Best Score x in course y.
        \end{enumerate}
    \item[b.] Every person who buys a policy is clever.
        $$ \forall \hspace{0.5em} x \hspace{0.5em} \forall \hspace{0.5em} p \hspace{0.5em}
        Buys(x, p) \rightarrow Clever(x) $$

    \item[c.] No person buys an expensive policy.
        $$ \forall \hspace{0.5em} x \hspace{0.5em} \forall \hspace{0.5em} p \hspace{0.5em}
        Expensive(p) \rightarrow \neg Buys(x, p) $$

    \item[d.] There is a barber who shaves all men in town who do not shave
        themselves.
        $$ \exists \hspace{0.5em} x \hspace{0.5em} Barber(x) \wedge
        \hspace{0.5em} \forall \hspace{0.5em} y \hspace{0.5em} \neg Shaves(y,
        y) \rightarrow Shaves(x, y) $$

    \item[e.] A person born in the UK, each of whose parents is a UK
        citizen or a UK resident, is a UK citizen by birth.
        $$ \forall \hspace{0.5em} x \hspace{0.5em} Born(x, UK) \wedge
        \hspace{0.5 em} \forall \hspace{0.5em} y \hspace{0.5em} Parent(x, y)
        \rightarrow BornCitizen(y, UK) \rightarrow CitizenByBirth(x, UK) $$

    \item[f.] A person born outside the UK, one of whose parents is a UK
        citizen by birth, is a UK citizen by descent.
        $$ \forall \hspace{0.5em} x \hspace{0.5em} \neg Born(x, UK) \wedge
        \hspace{0.5em} \exists \hspace{0.5em} y \hspace{0.5em} Parent(x, y)
        \wedge BornCitizen(y, UK) \rightarrow CitizenByDescent(x, UK) $$

    \item[g.] Politicians can fool some of the people all of the time and all
        of the people some of the time, but they cannot fool all of the
        of the people all of the time.
        $$ \forall \hspace{0.5em} x \hspace{0.5em} Politician(x) \wedge
        \hspace{0.5em} \exists \hspace{0.5em} y \hspace{0.5em} \forall \hspace{0.5em} t
        \hspace{0.5em} canFool(x, y, t) $$
        $$ \forall \hspace{0.5em} x \hspace{0.5em} Politician(x) \wedge
        \hspace{0.5em} \forall \hspace{0.5em} y \hspace{0.5em} \exists \hspace{0.5em} t
        \hspace{0.5em} canFool(x, y, t) $$
        $$ \forall \hspace{0.5em} x \hspace{0.5em} Politician(x) \wedge
        \hspace{0.5em} \forall \hspace{0.5em} y \hspace{0.5em} \forall \hspace{0.5em} t
        \hspace{0.5em} \neg canFool(x, y, t) $$

    \item[h.] Some students took Russian in spring 2001.
        $$ \neg \forall \hspace{0.5em} x \hspace{0.5em} Student(x) \wedge
        Russian(x, 2001, Spring) $$

    \item[i.] Only some students took German in spring 2001.
        $$ \neg \forall \hspace{0.5em} x \hspace{0.5em} Student(x) \wedge
        German(x, 2001, Spring) $$

    \item[j.] The best score in German is always higher than the best score
        in Russian.
        $$ \forall \hspace{0.5em} x \hspace{0.5em} \forall \hspace{0.5em} y
        \hspace{0.5em} BestScore(x, German) > BestScore(y, Russian) $$

\end{enumerate}

\rule[0.1pt]{40em}{1.0pt}

\section*{3. Unifying Pairs of Expression}  Attempt to unify the following pairs
of expressions.  Either show their most general unifiers, or explain why they
will not unify. Use the ${a/b}$ substitution form. Upper case letters are
variables, lowercase are constants. Assume all variables are universally
instantiated. 

\begin{enumerate}

    \item[a.] $p(X, a, Y)$ and $p(Z, Z, b)$ \par
        Substitution Set: $\{X/a, Y/b, Z/a\}$\par
        Result Expression: $p(a, a, b)$
               
    \item[b.] $p(X, X)$ and $p(a, b)$ \par
        It is not possible to unify these expressions because to unify them,
        the substitution set would have to contain two different values for
        variable X.

    \item[c.] $p(X, Y, Z)$ and $p(c, d, d)$\par
        Substitution Set: $\{X/c, Y/d, Z/d\}$\par
        Result Expression: $p(c, d, d)$

    \item[d.] $ancestor(X, father(X))$ and $ancestor(david, george)$ \par
        If we assume $father(david) = george$ then we can unify these
        expressions \par
                par
                                Substitution Set: $\{X/david\}$\par
        Result Expression: $ancestor(david, father(david))$

    \item[e.] $p(a, X)$ and $p(Y,Z)$ \par
        Substitution Set: $ \{X/Z, Y/a\} $

\end{enumerate}

\rule[0.1pt]{40em}{1.0pt}

\section*{4. Knowledge Base} Consider the following knowledge base:

\begin{enumerate}

    \item[a.] Prove that $Q$ is true with:

        \begin{center}
            \begin{minipage}{0.4\textwidth}
                \begin{enumerate}
                    \item[1.] $P \rightarrow Q$
                    \item[2.] $L \land M \rightarrow P$
                    \item[3.] $B \land L \rightarrow M$
                    \item[4.] $A \land P \rightarrow L$
                    \item[5.] $A \land B \rightarrow L$
                    \item[6.] A
                    \item[7.] B
                \end{enumerate}
            \end{minipage}
        \end{center}

        \begin{enumerate}
            \item[i.] Forward-Chaining \\
                If we assume that A and B are true, then:
                $$ A \wedge B \rightarrow L $$
                $$ B \wedge L \rightarrow M $$
                $$ L \wedge M \rightarrow P $$
                $$ P \rightarrow Q $$
            \item[ii.] Backward-Chaining
                $$ Q $$
                $$ P \rightarrow Q $$
                $$ L \wedge M \rightarrow P $$
                $$ B \wedge L \rightarrow M $$
                $$ A \wedge B \rightarrow L $$
                $$ A $$
                $$ B $$

            \item[iii.] Resolution \par
                Remove implications
                \begin{center}
                    \begin{minipage}{0.4\textwidth}
                        \begin{enumerate}
                            \item[8..] $\neg P \vee Q$
                            \item[9.] $\neg L \vee \neg M \vee P$
                            \item[10.] $\neg B \vee \neg L \vee M$
                            \item[11.] $\neg A \vee \neg P \vee L$
                            \item[12.] $\neg A \vee \neg B \vee L$
                        \end{enumerate}
                    \end{minipage}
                \end{center}
                Resolutions

        \end{enumerate}

    \item[b.] Prove $t \rightarrow s$

        \begin{center}
            \begin{minipage}{0.4\textwidth}
                \begin{enumerate}
                    \item[1.] $p \rightarrow q$
                    \item[2.] $[q \land r] \rightarrow s$
                    \item[3.] $[t \land u] \rightarrow r$
                    \item[4.] $u \rightarrow w$
                    \item[5.] $t \rightarrow y$
                    \item[6.] $y \rightarrow u$
                    \item[7.] $r \rightarrow p$
                    \item[8.] $p \rightarrow m$
                \end{enumerate}
            \end{minipage}
        \end{center}

        \begin{enumerate}
            \item[i.] Express in clause form
            \item[ii.] Forward-Chaining\\
                If we assume that $t$ is true, then:
                $$ t \rightarrow y $$
                $$ y \rightarrow u $$
                $$ [t \land u] \rightarrow r $$
                $$ r \rightarrow p $$
                $$ p \rightarrow q $$
                $$ [q \land r] \rightarrow s $$

            \item[iii.] Backward-Chaining
                $$ s $$
                $$ [q \land r] \rightarrow s $$
                $$ p \rightarrow q $$
                $$ r \rightarrow p $$
                $$ [t \land u] \rightarrow r $$
                $$ y \rightarrow u $$
                $$ t \rightarrow y $$
                $$ t $$

            \item[iv.] Resolution
                \begin{center}
                    \begin{minipage}{0.4\textwidth}
                        \begin{enumerate}
                            \item[9.] $\neg p \vee q$
                            \item[10.] $\neg q \vee \neg r \vee s$
                            \item[11.] $\neg t \vee \neg u \vee r$
                            \item[12.] $\neg u \vee w$
                            \item[13.] $\neg t \vee y$
                            \item[14.] $\neg y \vee u$
                            \item[15.] $\neg r \vee p$
                            \item[16.] $\neg p \vee m$
                        \end{enumerate}
                    \end{minipage}
                \end{center}

        \end{enumerate}

\end{enumerate}

\rule[0.1pt]{40em}{1.0pt}

\section*{5. Stories}
\begin{enumerate}

    \item[a.] “All dogs who are not tired and are smart are happy.  Those dogs
        who do tricks are not stupid.  Fido can do tricks and is full of
        energy.  Happy dogs have exciting lives.”

        \begin{enumerate}
            \item[i.] translate the sentences of into predicate form

                \begin{enumerate}
                    \item[1.] $Dog(x)$ = x is a dog
                    \item[2.] $Tired(x)$ = x is tired
                    \item[3.] $Smart(x)$ = x is smart
                    \item[4.] $Happy(x)$ = x is happy
                    \item[5.] $Trick(x)$ = x can do tricks
                    \item[6.] $Stupid(x)$ = x is Stupid
                    \item[7.] $Exciting(x)$ = x has an exciting life
                    \item[8.] $Energy(x)$ = x is full of energy
                \end{enumerate}

            \item[ii.] transform the predicate sentences into clause form

                \begin{enumerate}
                    \item[1.] $\forall \hspace{0.5em} x \hspace{0.5em}
                        (Dog(x) \wedge \neg Tired(x) \wedge Smart(x) \rightarrow
                        Happy(x))$
                    \item[2.] $\forall \hspace{0.5em} x \hspace{0.5em}
                        (Dog(x) \wedge Trick(x) \rightarrow \neg Stupid(x))$
                    \item[3.] $Dog(fido) \wedge Trick(fido) \wedge Energy(fido)$
                    \item[4.] $Happy(x) \rightarrow Exciting(x)$
                \end{enumerate}

            \item[iii.] prove via forward chaining that fido has an exciting life
                $$ Dog(fido) \wedge Trick(fido) \wedge Energy(fido) $$
                $$ Dog(fido) \wedge Energy(fido) \rightarrow \neg Tired(fido) $$
                $$ Dog(fido) \wedge Trick(fido) \rightarrow Smart(fido) $$
                $$ Dog(fido) \wedge \neg Tired(fido) \wedge Smart(fido) \rightarrow
                    Happy(fido) $$
                $$ Happy(fido) \rightarrow Exciting(fido) $$

            \item[iv.] prove via backward chaining that fido has an exciting life
                $$ Exciting(fido) $$
                $$ Happy(fido) \rightarrow Exciting(fido) $$
                $$ Dog(fido) \wedge \neg Tired(fido) \wedge Smart(fido) \rightarrow
                    Happy(fido) $$
                $$ Dog(fido) \wedge Trick(fido) \rightarrow Smart(fido) $$
                $$ Dog(fido) \wedge Energy(fido) \rightarrow \neg Tired(fido) $$
                $$ Dog(fido) \wedge Trick(fido) \wedge Energy(fido) $$

            \item[v.] prove via resolution that fido has an exciting life\par
                Remove implications:
                \begin{center}
                    \begin{minipage}{0.4\textwidth}
                        \begin{enumerate}
                            \item[9.] $\neg Dog(x) \vee \neg Trick(x) \vee \neg
                                Energy(x)$
                            \item[10.] $\neg Dog(x) \vee \neg Trick(x) \vee
                                Smart(x)$
                            \item[11.] $\neg Dog(x) \vee Tired(x) \vee \neg
                                Energy(x)$
                            \item[12.] $\neg Dog(x) \vee \neg Tired(x) \vee
                                Smart(x)$
                            \item[13.] $\neg Dog(x) \vee \neg Tired(x) \vee
                                \neg Smart(x) \vee Happy(x)$
                            \item[14.] $\neg Happy(x) \vee Exciting(x)$
                        \end{enumerate}
                    \end{minipage}
                \end{center}
        \end{enumerate}

    \item[b.] “Anyone passing the history exams and winning the lottery is
        happy.  But anyone who studies or is lucky can pass all the exams.
        John did not study, but he is lucky.  Anyone who is lucky wins the
        lottery.”  

        \begin{enumerate}
            \item[i.] translate the sentences of into predicate form
                \begin{enumerate}
                    \item[1.] $History(x)$ = x passed the history exams
                    \item[2.] $Lottery(x)$ = x won the lottery
                    \item[3.] $Happy(x)$ = x is happy
                    \item[4.] $Study(x)$ = x studied
                    \item[5.] $Lucky(x)$ = x is lucky
                    \item[6.] $Exams(x)$ = x passed all the exams
                \end{enumerate}

            \item[ii.] transform the predicate sentences into clause form
                \begin{enumerate}
                    \item[1.] $\forall \hspace{0.5em} x \hspace{0.5em}
                        (History(x) \wedge Lottery(x) \rightarrow Happy(x))$
                    \item[2.] $\forall \hspace{0.5em} x \hspace{0.5em}
                        (Exams(x) \rightarrow (History(x))$
                    \item[3.] $\forall \hspace{0.5em} x \hspace{0.5em}
                        (Study(x) \vee Lucky(x) \rightarrow Exams(x))$
                    \item[4.] $\forall \hspace{0.5em} x \hspace{0.5em}
                        (Lucky(x) \rightarrow Lottery(x))$
                \end{enumerate}

            \item[iii.] prove via forward chaining that John is happy
                $$ \neg Study(john) \wedge Lucky(john) $$
                $$ Lucky(john) \rightarrow Exams(john) $$
                $$ Lucky(john) \rightarrow Lottery(john) $$
                $$ Exams(john) \rightarrow History(john) $$
                $$ History(john) \wedge Lottery(john) \rightarrow Happy(john) $$

            \item[iv.] prove via backward chaining that John is happy
                $$ Happy(john) $$
                $$ History(john) \wedge Lottery(john) \rightarrow Happy(john) $$
                $$ Exams(john) \rightarrow History(john) $$
                $$ Lucky(john) \rightarrow Lottery(john) $$
                $$ Lucky(john) \rightarrow Exams(john) $$
                $$ Lucky(john) $$

            \item[v.] prove via resolution that John is happy\par
                Remove implications:
                \begin{center}
                    \begin{minipage}{0.4\textwidth}
                        \begin{enumerate}
                            \item[7.] $\neg Study(x) \vee \neg Lucky(x)$
                            \item[8.] $\neg Lucky(x) \vee Exams(x)$
                            \item[9.] $\neg Lucky(x) \vee Lottery(x)$
                            \item[10.] $\neg Exams(x) \vee History(x)$
                            \item[11.] $\neg History(x) \vee \neg Lottery(x) \vee
                                Happy(x)$
                        \end{enumerate}
                    \end{minipage}
                \end{center}

        \end{enumerate}

\end{enumerate}

\end{flushleft}
\end{document}
