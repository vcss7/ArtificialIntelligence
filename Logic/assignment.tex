\documentclass[11pt]{article}

\usepackage{xcolor}
\usepackage{enumitem}
\usepackage[margin=1.0in]{geometry}
\usepackage[none]{hyphenat}
\usepackage{amsfonts,amsmath,amssymb}
\usepackage{fancyhdr}

\thispagestyle{fancy}
\headheight14pt
\fancyhead{}
\fancyhead[L]{\slshape\MakeUppercase{\large Assignment-02: Logic and Logic
    Reasoning}} 
\fancyhead[R]{\large \slshape Victor Solis | \today}

\begin{document}
\begin{flushleft}
Artificial intelligence assignment 2; logic and logic reasoning.

\section*{1. Logic}

Consider the following statements:
    \begin{enumerate}

        \item[a.] $((smoke \land heat \implies) (smoke \implies fire) \lor (heat \implies fire))$
            \begin{enumerate} 

                \item[i.] Show the truth table for this sentence.
                \item[ii.] Is this sentence Valid, Satisfiable, or neither?

            \end{enumerate}
        \item[b.] $(a_1 \lor a_3) \land (\neg a_1 \lor a_2) \land (\neg a_1 \lor a_4) \land (\neg a_1 \lor \neg a_4) \land (\neg a_3)$

            \begin{enumerate} 
                \item[i.] Show the truth table for this sentence.
                \item[ii.] Is this sentence Valid, Satisfiable, or neither?
            \end{enumerate}

    \end{enumerate}

\rule[0.1pt]{40em}{1.0pt}

\section*{2. Vocabulary} First-order logic with consistent vocabulary

Represent the following sentences in first-order logic, using a consistent
    vocabulary (which you must define as well).

\begin{enumerate}

    \item[a.] Define the vocabulary. \textbf{For example}, you might have:
        \begin{enumerate} 
            \item[i.] Buys(X, Y, Z) = Person X buys Item Y from Person Z.
            \item[ii.] W > Score(X, Y, Z) = W is larger than score Person X
                gets in course Y during semester Z.
            \item[iii.] Predicates: Person(X), Clever(Y), Politician(Z)
        \end{enumerate}
    \item[b.] Every person who buys a policy is clever.
    \item[c.] No person buys an expensive policy.
    \item[d.] There is a barber who shaves all men in town who do not shave
        themselves.
    \item[e.] A person born in the UK, each of whose parents is a UK
        citizen or a UK resident, is a UK citizen by birth.
    \item[f.] A person born outside the UK, one of whose parents is a UK
        citizen by birth, is a UK citizen by descent.
    \item[g.] Politicians can fool some of the people all of the time, and
        they can fool all of the people some of the time, but they can't
        fool all of the people all of the time.
    \item[h.] Some students took Russian in spring 2001.
    \item[i.] Only some students took German in spring 2001.
    \item[j.] The best score in German is always higher than the best score
        in Russian.

\end{enumerate}

\rule[0.1pt]{40em}{1.0pt}

\section*{3. Unifying Pairs of Expression}  Attempt to unify the following pairs
of expressions.  Either show their most general unifiers, or explain why they
will not unify. Use the ${a/b}$ substitution form. Upper case letters are
variables, lowercase are constants. Assume all variables are universally
instantiated. 

\begin{enumerate}

    \item[a.] $p(X, a, Y)$ and $p(Z, Z, b)$
    \item[b.] $p(X, X)$ and $p(a, b)$
    \item[c.] $p(X, Y, Z)$ and $p(c, d, d)$
    \item[d.] $ancestor(X, father(X))$ and $ancestor(david, george)$
    \item[e.] $p(a, X)$ and $p(Y,Z)$

\end{enumerate}

\rule[0.1pt]{40em}{1.0pt}

\section*{4. Knowledge Base} Consider the following knowledge base:

\begin{enumerate}

    \item[a.] Prove that $Q$ is true with:

        \begin{enumerate}
            \item[1.] $P \implies Q$
            \item[2.] $L \land M \implies P$
            \item[3.] $B \land L \implies M$
            \item[4.] $A \land P \implies L$
            \item[5.] $A \land B \implies L$
            \item[6.] A
            \item[7.] B
        \end{enumerate}

        \begin{enumerate}
            \item[i.] Forward-Chaining
            \item[ii.] Backward-Chaining
            \item[iii.] Resolution
        \end{enumerate}

    \item[b.] Prove $t \implies s$

        \begin{enumerate}
            \item[1.] $p \implies q$
            \item[2.] $[q \land r] \implies s$
            \item[3.] $[t \land u] \implies r$
            \item[4.] $u \implies w$
            \item[5.] $t \implies y$
            \item[6.] $y \implies u$
            \item[7.] $r \implies p$
            \item[8.] $p \implies m$
        \end{enumerate}

        \begin{enumerate}
            \item[i.] Express in clause form
            \item[ii.] Forward-Chaining
            \item[iii.] Backward-Chaining
            \item[iv.] Resolution
        \end{enumerate}

\end{enumerate}

\rule[0.1pt]{40em}{1.0pt}

\section*{5. Stories}
\begin{enumerate}

    \item[a.] “All dogs who are not tired and are smart are happy.  Those dogs
        who do tricks are not stupid.  Fido can do tricks and is full of
        energy.  Happy dogs have exciting lives.”

        \begin{enumerate}
            \item[i.] translate the sentences of into predicate form
            \item[ii.] transform the predicate sentences into clause form
            \item[iii.] prove via forward chaining that fido has an exciting life
            \item[iv.] prove via backward chaining that fido has an exciting life
            \item[v.] prove via resolution that fido has an exciting life
        \end{enumerate}

    \item[b.] “Anyone passing the history exams and winning the lottery is
        happy.  But anyone who studies or is lucky can pass all the exams.
        John did not study, but he is lucky.  Anyone who is lucky wins the
        lottery.”  

        \begin{enumerate}
            \item[i.] translate the sentences of into predicate form
            \item[ii.] transform the predicate sentences into clause form
            \item[iii.] prove via forward chaining that John is happy
            \item[iv.] prove via backward chaining that John is happy
            \item[v.] prove via resolution that John is happy
        \end{enumerate}

\end{enumerate}

\end{flushleft}
\end{document}
