\documentclass[11pt]{article}

\usepackage{xcolor}
\usepackage{enumitem}
\usepackage[margin=1.0in]{geometry}
\usepackage[none]{hyphenat}
\usepackage{amsfonts,amsmath,amssymb}
\usepackage{fancyhdr}

\thispagestyle{fancy}
\headheight14pt
\fancyhead{}
\fancyhead[L]{\slshape\MakeUppercase{\large Assignment-01: Search}}
\fancyhead[R]{\large \slshape Victor Solis | \today}

\begin{document}
\begin{flushleft}
Artificial intelligence assignment 1; practicing basic and informed search 
algorithms.

\section*{Game of Chess}

\subsection*{Problem}
Consider the game of Chess (https://en.wikipedia.org/wiki/Chess). Approximately,
how many different states can a chess game have? You may assume

\begin{enumerate}
    \item We do not consider pawn promotion
    \item We consider each pawn is a distinct piece
    \item We consider only states with no captured pieces, i.e. all states will
        have 32 pieces
    \item We do not consider pawn promotion, castling, or any other special
        cases.
\end{enumerate}

State any other assumptions made and show your work. Your answer does not have 
to be mathematically correct, but should be in the right magnitude for the 
assumptions you have made and make sense based on your math.

\subsection*{Answer}

Additional assumtions will be:
\begin{enumerate}
    \item Pawns can not exist in the 1\textsuperscript{st} or 
        8\textsuperscript{th} rank
    \item Pawns can not stack (i.e. two of the same colored pawns cannot be on
        the same column)
    \item Black/White have one bishop that resides in the black tiles and
        one bishop that resides on the white tiles
    \item How the position came to be is not relevent
    \item Kings can be within one tile of eachother
\end{enumerate}
We will start by calculating the number of possible positions of the pieces
which do not have tile restrictions. Then, we will place the pieces that do have
possible restriction; that is, the bishops and pawns.

First we start with a single rook. Without any other pieces on the board, it has
64 possible positions it can exists on.

Then, we go to a second rook. Since one of the tiles is occupied by the first
placed rook, the second rook only has 63 possible tiles ot exist on.

Similarilly, the third rook will only have 62 tiles that it can possible exist
on.

At this stage, we can recognize a pattern: As tiles get occupied by previous
    pieces, the number of available tiles reduces by one.

The number of states the pieces (minus the pawns and bishops) can be represented
by the following product series.
\begin{equation}
    s_{tot1} = \prod_{i = 0}^{p - 1} (64 - i)
\end{equation}
Where $s_{tot1}$ are the total number of possible states for the pieces without
tiles restrictions and $p$ is the number of pieces placed.

There are a total of 12 chess pieces without the pawns and bishops.
Therefore,
\begin{equation}
    s_{tot1} = \prod_{i = 0}^{11} (64 - i) = 64 * 63 * ... * 53 \approx 1.573 *
    10^{21}
\end{equation}


\hspace{1cm}
\hrule
\section*{Wolf, Goat, and Cabbage Problem}

\subsection*{Problem}
A farmer has just bought a wolf, a goat, and a cabbage. They must cross the
river to get home, but their small rowboat can only carry the farmer and 1
animal or vegetable at a time. However, the farmer cannot leave the goat
unsupervised with the cabbage, or the goat will eat the cabbage. Similarly,
the farmer cannot leave the goat unsupervised with the wolf, or the wolf
will eat the goat. Draw the full search space starting from the initial
state below. Indicate with different colors or labels which states would
not be further explored because something gets eaten, as well as which
states would not be further explored because they are repeats of previous
states (to avoid cycles)
%%% TODO: Add Initial State and Goal State inage

\subsection*{Answer}

\hspace{1cm}
\hrule
\section*{Search Algorithms on a State Space Graph}

\subsection*{Problem}
Consider the following state space graph with Initial State S and Goal State G

\begin{enumerate}
    \item Draw out the complete search tree, ignoring cycles by not allowing
        the same node in the path more than once. The tree has been started
        below to show formatting.
    \item List the Node exploration order for a Breadth First Search. For
        grading simplicity, when there is a tie, explore the nodes
        alphabetically.
    \item List the Node exploration order for a Depth First Search. For grading
        simplicity, when there is a tie, explore the nodes alphabetically.
    \item The table below shows the estimated distance from each node to the
        goal. Using these as the heuristic h(n), list the Node exploration
        order for a Best-First Search. For grading simplicity, when there is a
        tie, explore the
    \item Using the same table as the heuristic h(n), list the A* Node
        exploration order. For grading simplicity, when there is a tie, explore
        the nodes alphabetically. nodes alphabetically.
\end{enumerate}

\subsection*{Answer}

\hspace{1cm}
\hrule
\section*{Search Trees on a Grid}

\subsection*{Poblem}
Consider the grid below with start state 50 and goal state 49:

Draw the search tree for the following algorithms a-d assuming: 
\begin{enumerate}
    \item Black squares are walls that cannot be passed through. 
    \item Break ties in the order N-E-S-W.
    \item Each step has a uniform cost.
    \item The heuristic is the Manhattan distance to the goal. (Don’t count
        black wall tiles, but for heuristic estimate, go through them)
\end{enumerate}

\subsection*{Answer}
\begin{enumerate}
    \item Breadth-First Search (5 layers)
    \item Depth-First Search
    \item Beam Search with W=2 (5 layers)
    \item A*
    \item What is the optimal path found by A*
\end{enumerate}

\end{flushleft}
\end{document}
