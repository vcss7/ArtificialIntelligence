\documentclass[11pt]{article}

\usepackage{xcolor}
\usepackage{enumerate}
\usepackage{enumitem}
\usepackage[margin=1.0in]{geometry}
\usepackage[none]{hyphenat}
\usepackage{amsfonts,amsmath,amssymb}
\usepackage{fancyhdr}
\usepackage{tikz}
\usetikzlibrary{shapes}

\thispagestyle{fancy}
\headheight14pt
\fancyhead{}
\fancyhead[L]{\slshape\MakeUppercase{\large Assignment-04}}
\fancyhead[R]{\large \slshape Victor Solis | \today}

\begin{document}
\begin{flushleft}
Assignment 4: Planning

\section*{Question 1}

Consider the following Crypt-arithmetic problems, where all letters represent a
different digit and the resulting sum is correct. \textbf{Write out all
variables, domains and constraints of the problem.}

    \begin{enumerate}[label=(\alph*)]
        \item SATURN + URANUS = PLANETS
            \begin{itemize}
                \item Variables: S, A, T, U, R, N, P, L.
                \item Domains: Each varialbe can have values: $\{0, 1, 2, 3, 4,
                    5, 6, 7, 8, 9\}$.
                \item Constraints:
                    \begin{enumerate}[label=\arabic*.]
                        \item The arithmetic equation given.
                        \item (Presumably) Each variable must have a unique
                            digit assign to it.
                    \end{enumerate}
            \end{itemize}
        \item YES + SEND + ME + MORE = MONEY
            \begin{itemize}
                \item Variables: Y, E, S, N, D, M, O, R.
                \item Domains: Each varialbe can have values: $\{0, 1, 2, 3, 4,
                    5, 6, 7, 8, 9\}$.
                \item Constraints:
                    \begin{enumerate}[label=\arabic*.]
                        \item The arithmetic equation given.
                        \item (Presumably) Each variable must have a unique
                            digit assign to it.
                    \end{enumerate}
            \end{itemize}
    \end{enumerate}

\rule[0.1pt]{40em}{1.0pt}

\section*{Question 2}
Consider the following set of edges between nodes. Find a coloring using colors
red, blue, and green such that no two adjacent nodes are assigned the same
color.

\begin{center}
$\{ (a, b),(a, d),(b, c),(b, d),(b, g),(c, g),(d, e),(d, f),(d, g),(f, g) \}$
\end{center}

% Node styles
\tikzstyle{none}=[]
\tikzstyle{circle}=[fill=white, draw=black, shape=circle]
\tikzstyle{red circle}=[fill=red, draw=black, shape=circle]
\tikzstyle{green circle}=[fill=green, draw=black, shape=circle]
\tikzstyle{blue circle}=[fill=blue, draw=black, shape=circle]

\begin{enumerate}[label=(\alph*)]
\itemsep1em

    \item Define a Constraint Satisfaction Problem for this problem. Clearly
        define the variables, domains, and constraints.

        \begin{itemize}
            \item Variables: The color of each of the nodes $\{a, b, c, d, e, f,
                g\}$
            \item Domains: For each node, the possible colors of the node. $\{
                    red, blue, green\}$
            \item Constraints: The edges specified must exist and if two nodes
                are connected by an edge, they cannot be the same color.
        \end{itemize}
        
    \item Draw the binary constraint graph for this Constraint Satisfaction
        Problem.

        \begin{center}
            \begin{tikzpicture}
                \node [style=circle] (1) at (0,0) {a};
                \node [style=circle] (2) at (-1, -0.75) {b};
                \node [style=circle] (3) at (-1.5, -1.75) {c};
                \node [style=circle] (4) at (1, -0.75) {d};
                \node [style=circle] (5) at (1.5, -1.75) {e};
                \node [style=circle] (6) at (0.5, -2.75) {f};
                \node [style=circle] (7) at (-0.5, -2.75) {g};
                \node [style=none] (8) at (0, 0.5) {\{red, blue, green\}};
                \node [style=none] (9) at (3, -0.5) {\{red, blue, green\}};
                \node [style=none] (14) at (-3, -0.5) {\{red, blue, green\}};
                \node [style=none] (13) at (3.5, -1.75) {\{red, blue, green\}};
                \node [style=none] (11) at (-3.5, -1.75) {\{red, blue, green\}};
                \node [style=none] (12) at (1.75, -3.5) {\{red, blue, green\}};
                \node [style=none] (10) at (-1.75, -3.5) {\{red, blue, green\}};
                \draw (1) to (2);
                \draw (1) to (4);
                \draw (2) to (3);
                \draw (2) to (4);
                \draw (2) to (7);
                \draw (3) to (7);
                \draw (4) to (7);
                \draw (4) to (5);
                \draw (6) to (7);
                \draw (6) to (5);
            \end{tikzpicture}
        \end{center}

        \begin{itemize}
            \item $C_a \neq C_b$
            \item $C_a \neq C_d$
            \item $C_b \neq C_c$
            \item $C_b \neq C_d$
            \item $C_b \neq C_g$
            \item $C_c \neq C_g$
            \item $C_d \neq C_e$
            \item $C_d \neq C_f$
            \item $C_d \neq C_g$
            \item $C_f \neq C_g$
        \end{itemize}

    \item Find at least one solution to the Constraint Satisfaction Problem.
        \begin{center}
            \begin{tikzpicture}
                \node [style=red circle] (1) at (0,0) {a};
                \node [style=blue circle] (2) at (-1, -0.75) {b};
                \node [style=green circle] (3) at (-1.5, -1.75) {c};
                \node [style=green circle] (4) at (1, -0.75) {d};
                \node [style=red circle] (5) at (1.5, -1.75) {e};
                \node [style=blue circle] (6) at (0.5, -2.75) {f};
                \node [style=red circle] (7) at (-0.5, -2.75) {g};
                \draw (1) to (2);
                \draw (1) to (4);
                \draw (2) to (3);
                \draw (2) to (4);
                \draw (2) to (7);
                \draw (3) to (7);
                \draw (4) to (7);
                \draw (4) to (5);
                \draw (6) to (7);
                \draw (6) to (5);
            \end{tikzpicture}
        \end{center}

\end{enumerate}
\rule[0.1pt]{40em}{1.0pt}

\section*{Question 3}
Consider a block stacking robot with the following actions:

\begin{itemize}
    \item[$\blacksquare$] Stack(x, y)
        \begin{itemize}
            \item Preconditions: Clear(y), Holding(x)
            \item Effects: armEmpty, On(x, y), ¬Clear(y), ¬Holding(x)
        \end{itemize}
    \item[$\blacksquare$] Unstack(x, y)
        \begin{itemize}
            \item Preconditions: Clear(x), On(x, y), armEmpty
            \item Effects: ¬armEmpty, ¬On(x, y), Clear(y), Holding(x)
        \end{itemize}
    \item[$\blacksquare$] Pickup(x)
        \begin{itemize}
            \item Preconditions: Clear(x), On(x, TABLE), armEmpty
            \item Effects: ¬armEmpty, ¬On(x, TABLE), Holding(x)
        \end{itemize}
    \item[$\blacksquare$] Putdown(x)
        \begin{itemize}
            \item Preconditions: Holding(x)
            \item Effects: armEmpty, On(x, TABLE), ¬Holding(x)
        \end{itemize}
\end{itemize}

Create a plan for each of the initial state/goal pairs below.
Assume armEmpty is in initial state and the table has infinite space.

\tikzstyle{block}=[fill=white, draw=black, shape=rectangle]
\tikzstyle{floor}=[-, fill=black]

\begin{enumerate}[label=(\alph*)]
\itemsep1em
    \item 
        \hspace{2em}
        Initial State:
        \begin{tikzpicture}
            \node [style=block] (2) at (1.25, -0.75) {A};
            \node [style=block] (3) at (0.75, -0.75) {B};
            \node [style=none] (4) at (0.25, -1) {};
            \node [style=none] (5) at (1.75, -1) {};
            \draw [style=floor] (4.center) to (5.center);
        \end{tikzpicture}
        \hspace{2em}
        Goal State:
        \begin{tikzpicture}
            \node [style=block] (0) at (3.5, -0.25) {A};
            \node [style=block] (1) at (3.5, -0.75) {B};
            \node [style=none] (6) at (2.75, -1) {};
            \node [style=none] (7) at (4.25, -1) {};
            \draw [style=floor] (6.center) to (7.center);
        \end{tikzpicture}

        \vspace{1em}Plan:
        \begin{enumerate}[label=\arabic*.]
            \item PickUp(A)
            \item Stack(A, B)
        \end{enumerate}

    \item
        \hspace{2em}
        Initial State:
        \begin{tikzpicture}
            \node [style=block] (0) at (1, -0.25) {A};
            \node [style=block] (1) at (1, -0.75) {B};
            \node [style=none] (4) at (0.25, -1) {};
            \node [style=none] (5) at (1.75, -1) {};
            \draw [style=floor] (4.center) to (5.center);
        \end{tikzpicture}
        \hspace{2em}
        Goal State:
        \begin{tikzpicture}
            \node [style=block] (2) at (3.5, -0.75) {A};
            \node [style=block] (3) at (3.5, -0.25) {B};
            \node [style=none] (6) at (2.75, -1) {};
            \node [style=none] (7) at (4.25, -1) {};
            \draw [style=floor] (6.center) to (7.center);
        \end{tikzpicture}

        \vspace{1em}Plan:
        \begin{enumerate}[label=\arabic*.]
            \item Unstack(A)
            \item Putdown(A)
            \item Pickup(B)
            \item Stack(B, A)
        \end{enumerate}

    \item
        \hspace{2em}
        Initial State:
        \begin{tikzpicture}
            \node [style=block] (8) at (0.75, -1.25) {A};
            \node [style=block] (9) at (1.25, -1.25) {B};
            \node [style=block] (10) at (0.75, -0.75) {C};
            \node [style=none] (4) at (0.25, -1.5) {};
            \node [style=none] (5) at (1.75, -1.5) {};
            \draw [style=floor] (4.center) to (5.center);
        \end{tikzpicture}
        \hspace{2em}
        Goal State:
        \begin{tikzpicture}
            \node [style=none] (6) at (2.75, -1.5) {};
            \node [style=none] (7) at (4.25, -1.5) {};
            \node [style=block] (11) at (3.5, -0.25) {A};
            \node [style=block] (12) at (3.5, -0.75) {B};
            \node [style=block] (13) at (3.5, -1.25) {C};
            \draw [style=floor] (6.center) to (7.center);
        \end{tikzpicture}

        \vspace{1em}Plan:
        \begin{enumerate}[label=\arabic*.]
            \item Unstack(C, A)
            \item Putdown(C)
            \item Pickup(B)
            \item Stack(B, C)
            \item Pickup(A)
            \item Stack(A, C)
        \end{enumerate}

\end{enumerate}

\rule[0.1pt]{40em}{1.0pt}

\section*{Question 4}
Consider the following simple planning problem in which the objective is to
interchange the values of two variables v1 and v2

\begin{itemize}
    \item Initial State: Value(v1, 3), Value(v2, 5), Value(v3, 0)
    \item Goal State: Value(v1, 5), Value(v2, 3)
    \item Actions:
        \begin{itemize}
            \item Assign(V, W, X, Y)
                \begin{itemize}
                    \item[$\blacksquare$] Preconditions: Value(V, X),
                        Value(W, Y)
                    \item[$\blacksquare$] Effects: Value(V,
                        Y), ¬Value(W, X)
                \end{itemize}
        \end{itemize}
\end{itemize}


\end{flushleft}
\end{document}
